\documentclass[ignorenonframetext,]{beamer}

\usepackage{hyperref}
\usepackage{soul}
\usepackage{minted}

\newcommand{\myurl}[1]{\textcolor{blue}{\underline{\url{#1}}}}

\usetheme{Warsaw}

% Various settings
\setbeamercovered{transparent} % Just gray covered text in overlays
\useoutertheme{infolines} % Display the slides number in the footline (and a smaller header)


% Add a title slide before each part, section and subsection
% TODO: I want round corners and a little shadow like on the title page :<
\AtBeginPart{
    \let\insertpartnumber\relax
    \let\partname\relax
    \frame{\partpage}
}
\AtBeginSection{
    \let\insertsectionnumber\relax
    \let\sectionname\relax
    \frame{\sectionpage}
}
\AtBeginSubsection{
    \let\insertsubsectionnumber\relax
    \let\subsectionname\relax
    \frame{\subsectionpage}
}

\setlength{\parindent}{0pt}
\setlength{\parskip}{6pt plus 2pt minus 1pt}
\setlength{\emergencystretch}{3em}  % prevent overfull lines
\setcounter{secnumdepth}{0}


\title{Introduction to Python scripting}
\author{Paul ``Dettorer'' Hervot}
\date{8 November 2014}

\begin{document}
\frame{\titlepage}

\section{Basics}\label{basics}
\subsection{Meta}\label{meta}

\begin{frame}{Classification}
    \begin{itemize}
        \item Imperative
        \item High level
        \item Interpreted
        \item Object oriented
        \item Functional style is easily doable!
    \end{itemize}
\end{frame}

\begin{frame}{Zen of Python (PEP 20)}

    Python Enhancement Proposal number 20

    \setbeamertemplate{itemize/enumerate body begin}{\tiny}
    \begin{itemize}
        \item<1-1> Beautiful is better than ugly.
        \item<1-2> Explicit is better than implicit.
        \item<1-2> Simple is better than complex.
        \item<1-1> Complex is better than complicated.
        \item<1-1> Flat is better than nested.
        \item<1-1> Sparse is better than dense.
        \item<1-2> Readability counts.
        \item<1-1> Special cases aren't special enough to break the rules.
        \item<1-1> Although practicality beats purity.
        \item<1-2> Errors should never pass silently.
        \item<1-1> Unless explicitly silenced.
        \item<1-1> In the face of ambiguity, refuse the temptation to guess.
        \item<1-2> There should be one -- and preferably only one -- obvious way to do it.
        \item<1-1> Although that way may not be obvious at first unless you're Dutch.
        \item<1-1> Now is better than never.
        \item<1-1> Although never is often better than \emph{right} now.
        \item<1-2> If the implementation is hard to explain, it's a bad idea.
        \item<1-1> If the implementation is easy to explain, it may be a good idea.
        \item<1-2> Namespaces are one honking great idea -- let's do more of those!
    \end{itemize}
    \setbeamertemplate{itemize/enumerate body begin}{\large}
\end{frame}

\begin{frame}{Zen of Python (PEP 20)}

    Python Enhancement Proposal number 20

    \begin{itemize}[<+-| alert@+>]
        \item Readability counts.
        \item Explicit is better than implicit.
        \item Simple is better than complex.
        \item Errors should never pass silently (unless explicitly silenced)
        \item There should be one -- and preferably only one -- obvious way to do it.
        \item If the implementation is hard to explain, it's a bad idea.
        \item Namespaces are one honking great idea, let's do more of those!
    \end{itemize}
\end{frame}

\begin{frame}%{Zen of Python (PEP 20)}

    For some examples, see

    \myurl{http://artifex.org/~hblanks/talks/2011/pep20_by_example.html}

\end{frame}

\subsection{Syntax}\label{basic-syntax}
\subsection{Data structures}\label{data-structures}

\section{Modules}\label{some-useful-modules}
\subsection{MANY exists}
\subsection{Write your own ones!}

\section{The end}
\subsection{A bit further}\label{a-bit-further}
\subsection{And now for something completely different}\label{and-now-for-something-completely-different}

\end{document}
