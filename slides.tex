\documentclass[ignorenonframetext,]{beamer}

\usepackage{hyperref}
\usepackage{soul}
\usepackage{minted}

\usepackage{tcolorbox}
\usepackage{etoolbox}
\BeforeBeginEnvironment{minted}{\begin{tcolorbox}}
\AfterEndEnvironment{minted}{\end{tcolorbox}}

\newcommand{\myurl}[1]{\textcolor{blue}{\underline{\url{#1}}}}

\usetheme{Warsaw}
\useoutertheme{infolines} % Display the slides number in the footline (and a smaller header)

% Various settings
\setbeamercovered{transparent} % Just gray covered text in overlays


% Add a tableofcontent slide before each subsection
\AtBeginSubsection{
    \frame{\tableofcontents[currentsubsection]}
}

\setlength{\parindent}{0pt}
\setlength{\parskip}{6pt plus 2pt minus 1pt}
\setlength{\emergencystretch}{3em}  % prevent overfull lines
\setcounter{secnumdepth}{0}


\title{Introduction to Python scripting}
\author{Paul ``Dettorer'' Hervot}
\date{8 November 2014}

\begin{document}
\frame{\titlepage}

\frame{\tableofcontents}

\section{Basics}
\subsection{Meta}

\begin{frame}{Classification}
    \begin{itemize}
        \item Imperative
        \item High level
        \item Interpreted
        \item Object oriented
        \item Functional style is easily doable!
    \end{itemize}
\end{frame}

\begin{frame}{Zen of Python (PEP 20)}

    Python Enhancement Proposal number 20

    \setbeamertemplate{itemize/enumerate body begin}{\tiny}
    \begin{itemize}
        \item<1-1> Beautiful is better than ugly.
        \item<1-2> Explicit is better than implicit.
        \item<1-2> Simple is better than complex.
        \item<1-1> Complex is better than complicated.
        \item<1-1> Flat is better than nested.
        \item<1-1> Sparse is better than dense.
        \item<1-2> Readability counts.
        \item<1-1> Special cases aren't special enough to break the rules.
        \item<1-1> Although practicality beats purity.
        \item<1-2> Errors should never pass silently.
        \item<1-1> Unless explicitly silenced.
        \item<1-1> In the face of ambiguity, refuse the temptation to guess.
        \item<1-2> There should be one -- and preferably only one -- obvious way to do it.
        \item<1-1> Although that way may not be obvious at first unless you're Dutch.
        \item<1-1> Now is better than never.
        \item<1-1> Although never is often better than \emph{right} now.
        \item<1-2> If the implementation is hard to explain, it's a bad idea.
        \item<1-1> If the implementation is easy to explain, it may be a good idea.
        \item<1-2> Namespaces are one honking great idea -- let's do more of those!
    \end{itemize}
\end{frame}

\begin{frame}{Zen of Python (PEP 20)}

    Python Enhancement Proposal number 20

    \begin{itemize}[<+-| alert@+>]
        \item Readability counts.
        \item Explicit is better than implicit.
        \item Simple is better than complex.
        \item Errors should never pass silently (unless explicitly silenced)
        \item There should be one -- and preferably only one -- obvious way to do it.
        \item If the implementation is hard to explain, it's a bad idea.
        \item Namespaces are one honking great idea, let's do more of those!
    \end{itemize}
\end{frame}

\begin{frame}%{Zen of Python (PEP 20)}

    For some examples, see

    \myurl{http://artifex.org/~hblanks/talks/2011/pep20_by_example.html}

\end{frame}

\begin{frame}{Python version}

    This talk is about {\LARGE Python} {\Huge 3}

    When you have the choice, choose Python 3

    Discussion on the subject: \myurl{https://wiki.python.org/moin/Python2orPython3}\\

    What's new in Python 3: \myurl{https://docs.python.org/3/whatsnew/3.0.html}

\end{frame}

\subsection{Syntax}

\begin{frame}[fragile]{Numbers}

    Let's fire up the interactive interpretor!

    \begin{minted}{python}
>>> 2 + 2
4
>>> 50 - 5 * 6
20
>>> # A division always returns a float
>>> (50 - 5 * 6) / 4
5.0
>>> 8 / 5
1.6
>>> 8 // 5
1
>>> 17 % 3 # Modulo
2
    \end{minted}
\end{frame}

\begin{frame}[fragile]{Strings}
    \begin{minted}{python}
>>> 'This parrot is no more!'
'This parrot is no more!'
    \end{minted}

    \pause
    \begin{minted}{python}
>>> "It has ceased to be!"
'It has ceased to be!'
    \end{minted}

    \pause
    \begin{minted}{python}
>>> '''It's expired and gone to meet its maker!
... This is a late parrot!'''
"It's expired and gone to meet its maker!\nThis is a late parrot!"
    \end{minted}

    \pause
    \begin{minted}{python}
>>> 'It\'s a stiff!'
"It's a stiff!"
    \end{minted}
\end{frame}

\begin{frame}[fragile]{Strings}
    \begin{minted}{python}
>>> 'This is an ex-parrot!'.upper()
'THIS IS AN EX-PARROT!'
    \end{minted}

    \pause
    \begin{minted}{python}
>>> 'A smile, ' + 'two bangs, ' + 'and a religion.'
'A smile, two bangs, and a religion.'
    \end{minted}

    \pause
    \begin{minted}{python}
>>> 'Gooooood night ' + 'ding' * 5
'Gooooood night dingdingdingdingding'
    \end{minted}
\end{frame}

\begin{frame}[fragile]{\st{Variables} Names}

    Dynamic typing, duck typing: just think about \emph{names}

    \begin{minted}{python}
>>> 20
20
>>> width = 20
>>> height = 5 * 9
>>> width * height
900
>>> width = '"largeur" in french'
>>> height = '"hauteur" in french'
>>> width + ' and ' + height
'"largeur" in french and "hauteur" in french'
    \end{minted}
\end{frame}

\begin{frame}[fragile]{\st{Variables} Names}

    But really. Python is strongly typed

    \begin{minted}{python}
>>> count = 3
>>> 'Then shalt thou count to ' + count + '.'
Traceback (most recent call last):
  File "<stdin>", line 1, in <module>
TypeError: Can't convert 'int' object to str implicitly
    \end{minted}

    \pause
    In this case, use format:
    \begin{minted}{python}
>>> count = 3
>>> 'Then shalt thou count to {}.'.format(count)
'Then shalt thou count to 3.'
    \end{minted}
\end{frame}

\subsection{Data structures}
\begin{frame}{Lists}
\end{frame}

\begin{frame}{Dictionaries}
\end{frame}

\subsection{Flow control}

\begin{frame}[fragile]{Functions}

    Let's forget the interpretor for now.

    \begin{minted}{python}
def say_hello():
    print('And now for something completely different')

say_hello()
    \end{minted}

Out:
    \begin{tcolorbox}
And now for something completely different
    \end{tcolorbox}
\end{frame}

\begin{frame}[fragile]{Functions}

    \begin{minted}{python}
def launch_holy_grenade(grenade):
    count = 3
    take_holy_pin_out(grenade)
    wait(count)
    throw(grenade)
    print('Boom!')

grenade = get_new_grenade()
launch_holy_grenade(grenade)
    \end{minted}

Out:
    \begin{tcolorbox}
Boom!
    \end{tcolorbox}
\end{frame}

\begin{frame}[fragile]{Loops}
    \begin{minted}{python}
>>> while i != 5:
...     i = random.randint(0,10)
    \end{minted}

    \pause
    \begin{minted}{python}
>>> range(3)
[0, 1, 2]
>>> for i in range(3):
...     print(i)
...
0
1
2
    \end{minted}
\end{frame}

\begin{frame}[fragile]{Loops}
    \begin{minted}{python}
>>> example_dict = {'first': 1, 'second': 2, 'third': 3}
>>> for key, value in example_dict.items():
...     print('key "{}" has value {}'.format(key, value))
...
key "second" has value 2
key "third" has value 3
key "first" has value 1
    \end{minted}
\end{frame}

\begin{frame}{Conditions}
\end{frame}

\begin{frame}[fragile]{Exception}
    \begin{minted}{python}
>>> count = 3
>>> try:
...     'Then shalt thou count to ' + count + '.'
... except TypeError:
...     'NI!'
...
'NI!'
    \end{minted}
\end{frame}

\section{Modules}
\subsection{MANY exist}
\subsection{Write your own ones!}

\section{The end}
\subsection{A bit further}
\subsection{And now for something completely different}

\end{document}
